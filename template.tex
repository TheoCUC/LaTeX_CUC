\documentclass{CUC}

\begin{document}
% 封面

% 中文摘要
\ZhAbstract{
    中国传媒大学非官方LaTeX本科毕业论文模板
}{
    姓名
}{
    随着3G时代的来临,移动通信技术正从提供简单的话音业务、低速数据业务向提供话音业务和可变带宽的多媒体数据业务相结合的方向发展。天线,作为移动通信无线链路中最重要的部件,也因此成为研究重点,受到了越来越广泛的关注。
}{
    数字信号处理,音频信号频谱,MATLAB,FFT,DSP
}
% 英文摘要
\EnAbstract{
    Unofficial LaTeX Undergraduate Thesis Template of CUC
}{
    Name
}{
    With the coming of the era named ‘Third Generation’, the mobile communication technologies are changing its direction from providing simple services of voice and low speed data transmission to providing combination services of voice and multi-media data transmission with variable band-width. Antennas, as the most important part in the radio link of mobile communication, become an important research topic, and attract more and more attentions.
}{
    DSP, Spectral Analysis, FFT, MATLAB
}
% 目录
\tableofcontents
% 正文
\Csection{绪 \quad 论}
\pagenumbering{arabic}
这部分内容我们来看PCA \cite{latextutorial}
\subsection{PCA的目的}
假设现在有这样一个情景\cite{zhengsongzhiZhuChengFenFenXiPCAJiQiMATLABDeShiXianFangFa2020}:现在要统计并可视化分析男大学生体测成绩,如果只参考立定跳远和1000m成绩两项指标,我们可以以立定跳远成绩作为$x$轴,1000m成绩作为$y$轴做出散点图,每个点代表一个学生;若统计三项指标,我们也可以在三维空间中做出散点图;但如果要统计四项乃至更多的指标,我们就无法再以此方法进行数据的可视化。
而主成分分析(Principal Component Analysis,PCA)的方法,可以将具有多个观测变量的高维数据集降维,使人们可以从事物之间错综复杂的关系中找出一些主要的方面,从而能更加有效地利用大量统计数据进行定量分析,并可以更好地进行可视化、回归等后续处理。
\subsection{PCA的几何意义}
先将问题简化为二维情形 \cite{yahzongQuerySimilarMusic2001}。有$N$个样品,具有两个观测变量$X_1,\ X_2$,做出散点图(如下图中的蓝色点),这样,在由$X_1,\ X_2$组成的坐标空间中,$N$个样品的分布情况如带状。现在问:如果现在要将两个观测变量缩减为一个,应该如何选取?
可以直观地看出,这$N$个样品无论沿$X_1$轴还是沿$X_2$轴方向,均有较大的离散性(其离散程度可以分别用观测变量$X_1$的方差和$X_2$的方差定量表示),也就是说,只考虑$X_1,\ X_2$的其中一个,原始数据均会有较大损失。
现在考虑以下线性组合\eqref{linearCombination},变换坐标空间:
\begin{equation}
    \begin{cases}
        T_1 = X_1 \cos \theta + X_2 \sin \theta \\
        T_2 = -X_1 \sin \theta + X_2 \cos \theta
    \end{cases} 
    \label{linearCombination}
\end{equation}
即
\begin{equation}
    \boldsymbol X = \boldsymbol U \boldsymbol \Sigma \boldsymbol V^* = \boldsymbol U \boldsymbol \Sigma \boldsymbol V^{\rm T}
\end{equation}
经过旋转,$N$个数据点在$T_1$轴上的离散程度最大,因而变量$T_1$代表了原始数据的绝大部分信息,这样,即使不考虑变量$T_2$也不会损失太多数据信息。这个$T_1$即为第一主成分(Principal Component 1,PC1),如图中箭头所示。若将所有数据点投影到$T_1$轴上(图中橙色点),就得到了降维后的数据。

\subsubsection{看看帅哥}
这部分芝士好无聊,不如来看一下我的照片,如图\ref{me}所示:
\begin{figure}[ht]
    \centering
    \includegraphics[width=0.25\textwidth]{img/zipai.jpeg}
    \caption{我的照片}
    \label{me}
\end{figure}

\subsubsection{高度近视同学别轻易买OLED手机}
另外要注意的一点是,OLED屏幕开启了DC调光功能后仍会有闪烁现象,并且闪烁频率降低,好处就是闪烁深度会大大降低。
因此,眼睛比较敏感的同学依然要尽量购买LCD手机,其中iPhone的LCD屏幕确实最为柔和,推荐如表\ref{iPhone}所示的几款,其中他们的二手市场价格也写在了表\ref{iPhone}中

\begin{table}[ht]
    \centering
    \caption{二手iPhone价格}
    \begin{tabular}{|c|c|c|}
        \hline
        iPhone SE2   &   iPhone Xr   &  iPhone 11   \\
        \hline
        2299         &  2899        &   3899    \\ 
        \hline
    \end{tabular}

    \label{iPhone}
\end{table}

\Csection{原理与步骤简述}
\subsection{算法一:特征分解(Eigen Decomposition)}
假设有一$n\times m$维的数据矩阵$\boldsymbol A = \begin{bmatrix} \vec a_1^{\rm T} \\ \vec a_2^{\rm T} \\ \vdots \\ \vec a_n^{\rm T}      \end{bmatrix}$,其中$n$为样本量,$m$为观测变量的数量。PCA的步骤如下:
\subsection{$r$的选取标准}
计算方差的累积贡献率:
\begin{equation}
  f(k)=\dfrac{\sum _{i=1}^i \lambda_k}{\sum_{i=1}^m \lambda_i},\quad k = 1,2,\cdots
  \label{another}
\end{equation}
作出其图像。因为$\lambda_1> \lambda_2 > \cdots$,故$f(k)$为一单调递增的函数,且其递增速度随着$k$增加逐渐降低。
一般地,我们可以取使得$f(r)\ge $某一阈值(如$95\%$)的最小的$r$,这样最多只会损失掉5\%的方差。
对于SVD法,将公式$\eqref{another}$中的$\lambda$换为$\sigma$即可。

% 参考文献
\clearpage
\bibliographystyle{gbt7714-numerical}
\bibliography{ref}
\addcontentsline{toc}{section}{参考文献}    % 把参考文献放入目录中

% 附录
\clearpage
\section*{附 \quad 录}
\addcontentsline{toc}{section}{附 \quad 录}

% 后记
\clearpage
\section*{后 \quad 记}
\addcontentsline{toc}{section}{后 \quad 记}
\end{document}